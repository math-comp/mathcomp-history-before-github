

\section{Introduction}


Recent works such as~\cite{avigad,fta,4color,flyspeck}
show that proof systems are getting sufficiently mature to formalise
non-trivial mathematical theories. Group theory is a domain
of mathematics where computer proofs could be of real added value.
This domain was one of the first to publish \emph{very long}
proofs. The first and most famous example is the 
Feit-Thompson theorem. Its historical proof~\cite{feit-thompson} is
255 pages long. That proof has later been simplified and
re-published~\cite{bender-glauberman,peterfalvi}, providing a better
understanding of local parts
of the proof. Yet its length remains unchanged, as well as its global
architecture.
Checking such a long proof with a computer would clearly
increase the confidence in its correctness, and hopefully lead to a
further step in the understanding of this proof.
This paper addresses the ground work needed to start formalising this
theorem.

There have been several attempts to formalise elementary group
theory using a proof assistant. Most of them~\cite{arthan,gunter1989,Yu90} 
stop at the Lagrange theorem.
An exception is Kamm�ller and Paulson~\cite{kam} who have formalised  
the first Sylow theorem. The originality of our work is that we do not use
elementary group theory as a mere example but as a foundation
for further formalisations. It is then crucial to us that our formalisation
scales up. We have therefore worked out a new development, with a
strong effort in proof engineering.

First of all, we reuse the \ssr{} extension of \Coq{}
developed by Gonthier for his proof of the \FCT{}.
This gives us a library and a proof language that is particularly
well suited to the formalisation of finite groups. Second, we make use
of many features of the \Coq{} proof engine (notations,
implicit arguments, coercions, canonical structures) to get more
readable statements and tractable proofs.


The paper is organised as follows. In Section 2, we present the 
\ssr{} extension and show how it is adequate to our needs.
In Section 3, we comment some of our choices in formalising
objects such as groups, quotients and morphisms. Finally, in Section 4,
we present some classic results of group theory that have already
been formally proved in this setting.

