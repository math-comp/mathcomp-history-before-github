\section{Conclusion}


To our knowledge, what is presented in this paper is already one of the most
complete formalisations of finite group theory. 
We cover almost all the material that can be found in an introductory course on 
group theory. Very few standard results like the simplicity
of the alternating group are still missing, but should be formalised very soon.
The only similar effort but in set theory
can be found in the Mizar system~\cite{Mizar}. Theorems like the ones
presented in Section 4 are missing from the Mizar formalisation.

We have deliberately specialised our formalisation to \emph{finite}
group theory: finite groups are \emph{not} obtained as a sub-class of
generic groups.
% Such a generic theory of groups is needed for the
% character theory part of the Feit-Thompson proof, hence we plan to
% elaborate this as a separated part of the development.
This design choice
is consistent with the usual presentation of group theory in the literature.
Imposing the inheritance of finite group
theory from generic group theory would be somehow artificial since they share
little results, all of them being trivial ones, hence it would not
help formalising proofs but create pointless delicate subtyping issues.  

Getting the definitions right is one of the most difficult aspects of
formalising mathematics. The problem is not much in capturing the 
semantics of each individual construct but rather in having all the concepts 
working together well. Group theory has been no exception in that respect.
It took much trial and error to arrive at the definitions 
presented in this paper. The fact that we were able to get 
results like the ones presented in Section 4 relatively
easily makes us confident 
that our base is robust enough to proceed to further formalisations.

Using \ssr{} has been a key aspect to our formal
development. Decidable types and a substantial use of rewriting for our proofs
give a `classical' flavour to our development that is more familiar
to what can be found in provers like {\sc Isabelle}~\cite{Isabelle} or
{\sc Hol}~\cite{Hol} than what is usually done in \Coq. Moreover, our novel use of canonical structures allows us to reconcile the convenience of
set-theoretic statements with the expressiveness of dependent types, by harnessing the automation power of type inference. We think that
this combination makes the \Coq{} system a powerful environment for
the formalisation of such algebraic theories.




An indication of the conciseness of our proof scripts
is given by the following figure. The standard library of \Coq{} contains
7000 objects (definitions + theorems) for  93000 lines of code, this makes a ratio of 
13 lines per object. The base library of \ssr{} plus our library for groups contains
1980 objects for 14400 lines, this makes a ratio of 7 lines per object.

