\section{Conclusion}


To our knowledge, what is presented in this paper is already one of the most
complete formalisation of finite group theory. 
We almost cover all the material that can be found in an introductory course on 
group theory. Very few standard results like the simplicity
of the alternating group are still missing, but should be formalised very soon.
The only similar effort but in set theory
can be found in the Mizar system~\cite{Mizar}. Theorems like the ones
presented in Section 4 are missing from the Mizar formalisation.

Getting the definitions right is one of the most difficult aspect of
formalising mathematics. The problem is not much in capturing the 
semantics of each individual construct but rather in having all the concepts 
working together well. Group theory has been no exception in that respect.
We had lots of try and go before converging to the definitions 
presented in this paper. The fact that we were able to get 
results like the ones presented in Section 4 relatively
easily makes us confident 
that our base is robust enough to proceed to further formalisations.

Using \ssr{} has been a key aspect to our formal development. It gives
us a very effective way of doing proofs inside the \Coq{} system.
Using decidable types and relying heavily on rewriting for our proofs
gives a `classical' flavour to our development that is more familiar
to what can be found in provers like {\sc Isabelle}~\cite{Isabelle} or {\sc Hol}~\cite{Hol} than 
what is usually done in \Coq.
An indication of the conciseness of our proof scripts
is given by the following figure. The standard library of \Coq{} contains
7000 objects (definitions + theorems) for  93000 lines of code, this makes a ratio of 
13 lines per object. The base library of \ssr{} plus our library for groups contains
1980 objects for 14400 lines, this makes a ratio of 7 lines per object.

