%%
%% This is file `squelette-rr.tex',
%% generated with the docstrip utility.
%%
%% The original source files were:
%%
%% RR.dtx  (with options: `sample')
%% ********************************************************************
%% Copyright (C) 1997-1999 2004 2006 INRIA/APICS
%% 
\documentclass[a4paper]{article}
\usepackage[microsoft]{RR}
\usepackage{hyperref}
\usepackage{url}
%\usepackage[latin1]{inputenc}
%\usepackage[T1]{fontenc}
\usepackage{amsfonts}
\usepackage{latexsym}
\usepackage{ae,aecompl,amsbsy,amssymb}
\usepackage{listings}

\newcommand{\Coq}{{\sc Coq}}
\newcommand{\ssr}{{\sc SSReflect}}
\newcommand{\C}[1]{\lstinline[basicstyle=\footnotesize][#1[}
\newcommand{\FCT}{Four Colour theorem}

\def\lstlanguagefiles{defSSR.tex}
\lstset{language=SSR}

%%\usepackage[frenchb]{babel} % optionnel
%%
%% date de publication du rapport
\RRdate{March 2007}
%%
%% Cas d'une version deux
%% \RRversion{2}
%% date de publication de la version 2
%% \RRdater{Novembre  2006}

%%
\RRauthor{Georges Gonthier\thanks{Microsoft Research, Cambridge, R-U}
\and
 Assia  Mahboubi\thanks{INRIA Futurs, Centre commun INRIA
   Microsoft Research}
\and Laurence Rideau\thanks[sfn]{INRIA Sophia Antipolis, projet Marelle}
\and Enrico Tassi\thanks{University of Bologna, Italy} 
\and Laurent Th\'ery\thanksref{sfn}}



% \RRauthor{% les auteurs
%  % Premier auteur, avec une note
% Georges Gonthier%\thanks{Footnote for first author}%
%   % note partag\'ee (optionnelle)
%   %\thanks[sfn]{Shared foot note}%
%  % \and entre chaque auteur s'il y en a plusieurs
%   \and
% Assia Mahboubi
% \and
% Enrico Tassi
% \and
% Laurent Th\'ery
% \and
% Laurence Rideau
 % r\'ef\'erence \`a la note partag\'ee
%\thanksref{sfn}{bla}
%%
%% Ceci apparait sur chaque page paire.
\authorhead{Gonthier \& al.}
%%
\RRtitle{Une formalisation modulaire des groupes finis}
%% English title
\RRetitle{A Modular Formalisation of Finite Group Theory}
%%
\titlehead{A Modular Formalisation of Finite Group Theory}

\RRresume{Ce rapport pr�sente une formalisation de th�orie des
  groupes �l�mentaires r�alis�e dans le syst�me \Coq{}. Ce travail est
  la premi�re �tape d'un projet d'envergure, qui a pour but de
  construire une preuve formelle du th�or�me de Feit-Thompson.
  Comme nos d�veloppements formels ult�rieurs reposeront de fa\c con
  cruciale sur cette base, nous avons consacr� un soin particulier �
  la modularit� de ces fondations.}

\RRabstract{In this paper, we present a formalisation of elementary group
theory done in {\sc Coq}. This work is the first milestone
of a long-term effort to formalise Feit-Thompson theorem.
As our further developments will heavily rely on this initial base,
we took special care to articulate it in the most
compositional way.}
%%
\RRmotcle{Groupes finis, assistants � la preuve, formalisation des
  math�matiques}

\RRkeyword{finite groups, proof assistants, formalisation of mathematics}
%%
%% \RRprojet{Apics}  % cas d'un seul projet
\RRprojets{Marelle, INRIA Sophia\\ Composants Math�matiques, Centre Commun
 INRIA Microsoft Research}
%%
%% \RRtheme{\THNum} % cas d'un seul theme
\RRtheme{\THSym} % cas de 5 themes
\URFuturs % pour ceux qui sont dans le virtuel
%\URSophia % pour ceux qui sont au Sud.
%%
\begin{document}
%%
\makeRR   % cas d'un rapport de recherche
%% \makeRT % cas d'un rapport technique.
%% a partir d'ici, chacun fait comme il le souhaite


\section{Introduction}


Recent work such as~\cite{avigad,fta,4color,flyspeck}
shows that proof systems are getting sufficiently mature to formalise
non-trivial mathematical theories. Group theory is a domain
of mathematics where computer proofs could be of real added value.
This domain was one of the first to publish \emph{very long}
proofs. The first and most famous example is the 
Feit-Thompson theorem. Its historical proof~\cite{feit-thompson} is
255 pages long. That proof has later been simplified and
re-published~\cite{bender-glauberman,peterfalvi}, providing a better
understanding of local parts
of the proof. Yet its length remains unchanged, as well as its global
architecture.
Checking such a long proof with a computer would clearly
increase the confidence in its correctness, and hopefully lead to a
further step in the understanding of this proof.
This paper addresses the ground work needed to start formalising this
theorem.

There have been several attempts to formalise elementary group
theory using a proof assistant. Most of them~\cite{arthan,gunter1989,Yu90} 
stop at the Lagrange theorem.
An exception is Kamm�ller and Paulson~\cite{kam} who have formalised  
the first Sylow theorem. The originality of our work is that we do not use
elementary group theory as a mere example but as a foundation
for further formalisations. It is then crucial to us that our formalisation
scales up. We have therefore worked out a new development, with a
strong effort in proof engineering.

First of all, we reuse the \ssr{} extension of \Coq{}
developed by Gonthier for his proof of the \FCT{}.
This gives us a library and a proof language that is particularly
well suited to the formalisation of finite groups. Second, we make use
of many features of the \Coq{} proof engine (notations,
implicit arguments, coercions, canonical structures) to get more
readable statements and tractable proofs.


The paper is organised as follows. In Section 2, we present the 
\ssr{} extension and show how it is adequate to our needs.
In Section 3, we comment some of our choices in formalising
objects such as groups, quotients and morphisms. Finally, in Section 4,
we present some classic results of group theory that have already
been formally proved in this setting.


\section{Small scale reflection}\label{sec:ssr}

The \ssr{} extension \cite{ssr} offers new syntax features for the
proof shell and a bunch of libraries making use of \emph{small scale
  reflection} in various respects. This layer above the standard
\Coq{} system provides a convenient framework for dealing with
structures equipped with a decidable equality. In this section, 
we comment the fundamental definitions present in the library and
how modularity is carried out throughout the development.

\subsection{Proof shell}\label{ssec:shell}

Proof scripts written with the \ssr{} extension have a very different
flavour than the ones developed using standard \Coq{} tactics. We are not 
going to present the proof shell extensively but only describe some 
simple features, that, we believe, have a positive impact on productivity.
A  script is a linear structure composed of tactics. Each tactic ends with
a period. 
An example of such a script is the following
\begin{lstlisting}
move=> x a H; apply: etrans (cardUI _ _).
  case: (a x); last by rewrite /= card0 card1.
  by rewrite [_ + x]addnC.
by rewrite {1}mem_filter /setI.
\end{lstlisting}
All the frequent bookkeeping operations that consists in moving, splitting, 
generalising formulas from (or to) the context are regrouped in 
a single tactic \C{move}, making these operations more intuitive.
For example, the fact that arguments of the \C{move} at the first line of the example of
script are after the arrow indicates that the three arguments are the name to associate
to three formulas to move from the conclusion to the context.

Good practise recommends to outline the underlying structure
of the proof by indenting. To further structure scripts, \ssr{}
first proposes a tactical \C{by}
to explicitly tag closing tactics. 
When replaying scripts, we then have the nice property that
an error immediately occurs when a closing tactic fails to prove its subgoal. 
Second, when composing tactics, 
the two tacticals \C{first} and \C{last} let the user restrict the application of
a tactic to only the first or the last subgoal generated by the previous
command. It covers the frequents cases where a tactic generates 
two subgoals one of which can be easily discarded.
In practice, these two tactics 
are so effective at increasing the linearity of our scripts that,
in fact, it is very rare than more than two levels of indentation are needed.

Finally, the \C{rewrite} tactic in \ssr{} comes with a concise syntax to accommodate
in a single command
all the possible combinations of conditional rewriting, unfolding of definition,
simplifying, rewriting selecting specific occurrences, rewriting selecting
specific patterns, to name only some of them. Rewriting is then really convivial 
and contributes to a change of proof style more based on equational reasoning. 
In the standard library of \Coq, the \C{rewrite} tactic is roughly used the same number of times
than the tactic  \C{apply}.
In our development for group theory, \C{rewrite} is used three times more than
\C{apply} --- despite the fact that, on average, each \ssr{} \C{rewrite}
stands for three \Coq{} \C{rewrite}s.

\subsection{Views}\label{ssec:ssr}

The \Coq{} system is based on an intuitionistic type theory, the Calculus
of Inductive Constructions \cite{werner,paulin}. There is 
a distinction between logical propositions and boolean values.
On the one hand, logical propositions are objects
of \emph{sort} \C{Prop}, which is the carrier of intuitionistic
reasoning. On the other hand, \C{bool} is an
inductive datatype with two constructors \C{true} and \C{false}, for
which the term \\
\quad\C{fun b -> if b returns (b \|\| negb b = true) then refl_equal true
else refl_equal true}\\
is a proof of \C{forall b: bool, b \|\| negb b = true}. 
This proof does a dependent case analysis on {\tt b} and
returns in each case a proof of \C{true = true}, the term
\C{(refl_equal true)}, thanks to the fact that boolean functions are
computable.

Boolean predicates of the Coq system support the preferred mode of
classical reasoning, which is the computation of truth values
tables. Reasoning in the \C{Prop} sort is mainly performed in a
natural deduction style.
 The small scale reflection proposes 
a generic mechanism to have the best of the two worlds and move freely 
from a propositional version of a
decidable predicate to its boolean version.
For this, booleans are injected into propositions 
using the coercion mechanism:
\begin{lstlisting}
Coercion is_true (b: bool) := b = true.
\end{lstlisting}
Now, every time the {\sc Coq} system expects a proposition but receives a boolean \C{b}, 
it will automatically coerce it into the proposition \C{(is_true b)}, i.e
the  proposition \C{b = true}. Coercions are also omitted by the prettyprinter,
so everything is completely transparent to the user.
Then, the inductive predicate \C{reflect}
is used to relate propositions and booleans
\begin{lstlisting}
Inductive reflect (P: Prop): bool -> Type :=
  |  Reflect_true: P => reflect P true
  | Reflect_false: ~P => reflect P false.
\end{lstlisting}
The statement \C{(reflect b P)} indicates that \C{(is\_true b)}
and \C{P} are two logically equivalent propositions. In the following, we use
the notation \C{b <-> P} for \C{(reflect b P)}.
For instance, the following lemma:
\begin{lstlisting}
Lemma andP: forall b$_1$ b$_2$, (b$_1$ /\ b$_2$) <-> (b$_1$ && b$_2$).
\end{lstlisting}
relates the boolean conjunction \C{\&\&} and
the logical one \lstinline[basicstyle=\footnotesize]+/\+. 
Note that in \C{andP}, \C{b$_1$} and \C{b$_2$} are two boolean variables and
the proposition \lstinline[basicstyle=\footnotesize]+b$_1$ /\ b$_2$+ hides two coercions.
The conjunction of \C{b$_1$} and \C{b$_2$} can then be viewed
as \C{b$_1$ /\\ b$_2$}  or as \C{b$_1$ && b$_2$}. 
A naming convention in \ssr{} is to postfix the name of view lemmas with \C{P}.
For example, \C{orP} relates  \C{\|\|} and \C{\\/}, \C{negP} relates  \C{negb} and \C{\~}.

Views are integrated to the proof language.
If we are to prove a goal of the form \C{(b$_1$ /\\ b$_2$) -> G},
the tactic \C{case => E$_1$  \ \ \  E$_2$} changes the goal to \C{G}
adding to the context the two assumptions
\C{E$_1$: b$_1$} and \C{E$_2$: b$_2$}.
If the goal is of the form \C{(b$_1$ \&\& b$_2$) -> G} instead, we simply need
to change the tactic to  \C{case/andP=> E$_1$ \ \ \ \  E$_2$} to perform
the necessary intermediate change of view.

Suppose now that our goal is \C{b$_1$ \&\& b$_2$}.
In order to split this goal into two subgoals, we use a combination of
two tactics: \C{apply/andP; split}.
The first tactic performs the change of view so that the 
second tactic can do the splitting. Note that if we happen to
have in the context an assumption \C{H: b$_1$},
instead of performing the splitting, the tactic
\C{rewrite H /=}, i.e., rewriting with \C{H} followed by  a simplification,
can directly be used to transform the goal \C{b$_1$ \&\& b$_2$} into \C{b$_2$}.


Views also provide a convenient way to swap between 
several (logical) characterisations of the same (computational)
definition, having  a view lemma per interpretation.
A trivial example is the ternary boolean conjunction.
If we have a goal of the form \lstinline[basicstyle=\footnotesize]+b$_1$ && (b$_2$ && b$_3$) -> G+,
applying the tactic \lstinline[basicstyle=\footnotesize]+case/andP+ leads to the
goal \lstinline[basicstyle=\footnotesize]+b$_1$ -> b$_2$ && b$_3$ -> G+.
We can also define an alternative view with
\begin{lstlisting}
Inductive and3 (P Q R: Prop): Prop := And3: P -> Q -> R -> (and3 P Q R).

Lemma and3P: forall b$_1$ b$_2$ b$_3$, (and3 b$_1$ b$_2$ b$_3$) <-> (b$_1$ && (b$_2$ && b$_3$)). 
\end{lstlisting}
Now, the tactic \lstinline[basicstyle=\footnotesize]+case/and3P+  directly transforms the goal
\lstinline[basicstyle=\footnotesize]+b$_1$ && (b$_2$ && b$_3$) -> G+ into \lstinline[basicstyle=\footnotesize]+b$_1$ -> b$_2$ -> b$_3$ -> G+.


\subsection{Libraries}\label{ssec:lib}


In our formalisation of finite groups, we reused the base libraries
initially  developed for the
formal proof of the \FCT . They consist in a hierarchy of
structures and a substantial toolbox to work with finite types.
At the bottom of this hierarchy, the structure \C{eqType} deals with
types with decidable equality.
\begin{lstlisting}
Structure eqType : Type := EqType {
    sort :> Type; 
  _ == _ :  sort -> sort -> bool; 
     eqP :  forall x y, (x = y) <-> (x == y)
}.
\end{lstlisting}
\noindent
The \C{:>} symbol declares \C{sort} as a coercion from an \C{eqType} to its
carrier type. It is the standard technique to get subtyping,
an object of type \C{eqType} can then be viewed as an object of type \C{Type}.
In the type theory of \Coq{}, the only relation we can rewrite with 
is the primitive (Leibniz) equality. When another
equivalence relation is the intended notion of equality on a given
type, the user usually needs to use the setoid
workaround \cite{BarCap}. Unfortunately, setoid rewriting does not have 
the same power as primitive rewriting.
An \C{eqType} structure not only assumes the existence of a decidable
equality \C{==} but also \C{eqP} injects this equality into the
Leibniz one, thus promoting it to a \emph{rewritable} relation.


Any non parametric inductive type can be turned
into an \C{eqType} choosing for \C{==} the function that
checks structural equality.
This is the case for booleans and natural numbers for which a \C{bool_eqType} 
and \C{nat_eqType} are defined as {\it canonical structures}. Canonical
structures are used when solving equations involving 
implicit arguments. Namely, if the type
checker needs to infer an \C{eqType} structure on the type
\C{nat}, it will automatically choose as a default choice the \C{nat\_eqType}
type. By enlarging the set of implicit arguments \Coq{}
can infer, canonical structures ease the handling of the
hierarchy of structures.

A key property of \C{eqType} structures is that
they enjoy proof-irrelevance for the
equality proofs of their elements: every equality proof is
convertible to a reflected boolean test.

\begin{lstlisting}
Lemma eq_irrelevance: forall (d: eqType) (x y: d) (E: x = y) (E': x = y), E = E'.
\end{lstlisting}

%and in particular for the \C{bool\_eqType} structure:
%
%\begin{lstlisting}
%Lemma bool_irrelevance : forall (x y : bool) (E E' : x = y), E = E'.
%\end{lstlisting}
%

An \C{eqType} structure only defines a domain, in which sets take their elements. Sets are 
then represented by their characteristic function 
\begin{lstlisting}
Definition set (d: eqType) := d -> bool.
\end{lstlisting}
and defining set operations like $\cup$ and $\cap$
is done by providing the corresponding boolean functions.

The next step consists in building lists, elements of type \C{seq d}, 
whose elements belong to the parametric \C{eqType} structure \C{d}. The decidability of equality
on \C{d} is needed when defining the basic operations on
lists like membership \C{\\in} and look-up \C{index}. Then, membership
is used for defining a coercion from list to set, such
that \C{(l x)} is automatically coerced into \C{x \\in l}.

Lists are the cornerstone of the definition of finite types. 
A \C{finType} structure is composed of a list of
elements of an \C{eqType} structure, each element being unique.

\begin{lstlisting}
Structure finType : Type := FinType {
  sort :> eqType;
  enum :  seq  sort;
  enumP :  forall x, count (set1 x) enum = 1
}.
\end{lstlisting}
where \C{(set1 x)} is the set that contains only \C{x} and 
\C{(count f l}) computes the number of elements
\C{y} of the list \C{l} for which \C{(f y)} is true.

Finite sets are then sets taken in a \C{finType} domain. In the library,
the basic operations are provided.
For example, given \C{A} a finite set, \C{(card A}) represents the cardinality of \C{A}.
All these operations come along with their basic properties. For example, we have:
\begin{lstlisting}
Lemma cardUI : forall (d: finType) (A B: set d), 
  card (A $\cup$ B) + card (A $\cap$ B) = card A + card B.
Lemma card_image : forall (d d': finType) (f: d -> d') (A: set d), 
     injective f => card (image f A) = card A.
\end{lstlisting}

\section{The group library}

This section is dedicated to the formalisation of elementary group
theory. We justify our definitions and explain how they relate to each other.

\subsection{Graphs of function and intensional sets}
We use the notation \C{f $=_1$ g} to indicate that
two functions are extensionally equal, i.e  the fact that
\C{forall x, f x = g x} holds. In \Coq, \C{f $=_1$ g}
does not implies \C{f = g}. This makes equational reasoning
with objects containing functions difficult in \Coq{}
without adding extra axioms. In our case, extra axioms are not needed.
The functions we manipulate have finite domain so they can be finitely
represented by their graph. Given \C{d$_1$} a finite type and \C{d$_2$}
a type with decidable equality, a graph is defined as

\begin{lstlisting}
Inductive fgraphType : Type := 
  Fgraph (val: seq d$_2$) (fgraph_sizeP: size val = card d$_1$): fgraphType.
\end{lstlisting}
It contains a list \C{val} of elements of \C{d$_2$}, the size of \C{val} being
exactly the cardinal of \C{d$_1$}. Defining
a function \C{fgraph\_of\_fun} that computes the graph associated to a function
is straightforward.
Conversely, a conversion \C{fun\_of\_fgraph} is defined to 
let the user manipulate graphs as standard functions.
With graphs as functions, it is possible to prove functional extensionality
\begin{lstlisting}
Lemma fgraphP : forall (f g : fgraphType d$_1$ d$_2$), f =$_1$ g $\Leftrightarrow$ f = g.
\end{lstlisting}
Note that on the left-hand side of the equivalence, \C{f =$_1$ g} is automatically
coerced into \C{(fun_of_graph f) =$_1$ (fun_of_graph g)}.
In order to make graphs a proper substitute to functions, we need
to equip them  with the same operations that  the ones proposed for 
functions. For example, \C{(setType d)} corresponds to \C{(set d)}.
We call elements of  \C{(setType d)} \emph{intensional} sets by opposition to the sets defined
by their characteristic function. The notation \C{\{x, f  x\}} is used
to define the intensional set whose characteristic function is \C{f}
and \C{(iimage f A)} corresponds to the intensional set of the image of \C{A} by \C{f}.

Graphs are used to build some useful datastructures. For example,
homogeneous tuples, i.e. sequences of elements of type \C{K} of fixed length \C{n}, 
are implemented as graphs with domain \C{(ordinal n)}, the finite type \C{\{0, 1, 2, $\dots$, n-1\}},
and co-domain \C{K}. With this representation, the
$n$-th element of a $p$-tuple $t$ can be obtained applying $t$ to $n$,
as soon as $n$ lies in the the domain of $t$. Also, permutations
are defined as function graphs with identical domain and co-domain,
the \C{val} list of which does not contain any duplicate.

\subsection{Groups}

In the same way than \C{eqType} structures were introduced before defining
sets, we introduce a notion of (finite) \emph{group domain} which is
distinct from the one of groups. It is modelled by a \C{finGroupType}
record structure
\begin{lstlisting}
Structure finGroupType : Type := FinGroupType {
  element :> finType;
        1 : element;
      _^-1 : element -> element;
    _ * _ : element -> element -> element;
    unitP : forall x, 1 * x = x;
     invP : forall x, x^-1 * x = 1;
    mulP : forall x$_1$ x$_2$ x$_3$,  x$_1$ * (x$_2$ * x$_3$) = (x$_1$ * x$_2$) * x$_3$
}.
\end{lstlisting}
It contains a carrier, a composition law
and an inverse function, a unit element and the usual properties of these
operations. Its first field is declared as a coercion to the carrier
of the group domain.

In the group library, a first category of lemmas is composed of 
properties that are valid on the whole group domain. For example:

\begin{lstlisting}
Lemma invg_mul : forall x$_1$ x$_2$, (x$_2$ * x$_1$)^-1 = x$_1$^-1 * x$_2$^-1. 
\end{lstlisting}

Also, we can already define operations on arbitrary sets of a group
domain. If \C{A} is such a set, we can define for instance:

\begin{lstlisting}
Definition   x ^ y  := y^-1 * x * y. 
Definition  A :* x  := {y, y * x^-1 \in A}. (* right cosets *)
Definition  A :^ x  := {y, y ^ x^-1 \in A}. (* conjugate *)
Definition normaliser A := {x, (A :^ x) $\subset$ A}.
\end{lstlisting}
Some definitions may look less intuitive at first sight since
we try as much as possible to define them as boolean
predicates. For example, the set of point-wise products of 
two sets is defined as:
\begin{lstlisting}
Definition A spp B := {xy, negb (disjoint {y, xy \in (A :* y)} B)}
\end{lstlisting}
\noindent A \emph{view} lemma gives the natural characterisation of this object:
\begin{lstlisting}
Lemma smulgP : forall A B z,  (exists x y, x \in A & y \in B & z = x * y) <-> (z \in A spp B).
\end{lstlisting}
Lemmas like \C{smulgP} belongs to the second category of lemmas 
composed of the properties of these operations requiring only group domain \emph{sets}.


Finally, a \emph{group} is defined as a boolean predicate, satisfied by
sets of a given group domain that contain the unit and are stable
under product.

\begin{lstlisting}
Definition group_set A := 1 \in A && (A spp A) $\subset$ A.\end{lstlisting}
It is very convenient to give the possibility of attaching in a
canonical way the proof that a set has a group structure. This is
why groups are declared as structures:

\begin{lstlisting}
Structure group(elt : finGroupType) : Type := Group {
  set_of_group :> setType elt;
  set_of_groupP : group_set set_of_group
}.
\end{lstlisting}
The first argument of this structure is a \emph{set}, giving the
carrier of the group. Notice that we do \emph{not} define one type per group 
but one type per group domain, which avoids having unnecessary injections
everywhere in the development.

Finally, the last category of lemmas in the library is composed of
group properties.  For example, given a group \C{H}, we have the following
property:

\begin{lstlisting}
Lemma groupMl : forall x y, x \in H => (x * y) \in H = y \in H.
\end{lstlisting}
In the above statement, the equality stands for \Coq{} standard equality
between boolean values, since membership of \C{H} is a boolean
predicate.


We declare a canonical group structure for the
usual group constructions so that they can
be displayed as their set carrier but still benefit from an automatically
inferred proof of group structure when needed.
For example, such canonical structure is defined for the intersection of two
groups \C{H} and \C{K} that share the group domain \C{elt}:
\begin{lstlisting}
Lemma group_setI : group_set (H $ \cap $ K).
Canonical Structure setI_group := Group group_setI.
\end{lstlisting}
\noindent where, as in the previous section, $\cap$ stands for the
\emph{set} intersection operation.

\noindent Given a group domain \lstinline[basicstyle=\footnotesize]+elt+ and two groups \lstinline[basicstyle=\footnotesize]+H+ and
\lstinline[basicstyle=\footnotesize]+K+, the stability of the group law for the intersection is proved in
the following way:
\begin{lstlisting}
Lemma setI_stable : forall x y, x \in (H $\cap$ K) => y \in (H $\cap$ K) => (x * y) \in (H $\cap$ K).
Proof. by move=> x y H1 H2; rewrite groupMl. Qed.
\end{lstlisting}
The group structure on the \lstinline[basicstyle=\footnotesize]+H $\cap$ K+ carrier is
automatically inferred from the
canonical structure declaration and the \lstinline[basicstyle=\footnotesize]+by+
closing command uses the \lstinline[basicstyle=\footnotesize]+H1+ and \lstinline[basicstyle=\footnotesize]+H2+
assumptions to close two trivial generated goals.

This two-level definition of groups, involving group
domain types and groups as first order citizens equipped with
canonical structures, plays an important
role in doing proofs. Type inference is then used to
perform proof inference from the database of registered canonical structures.

\subsection{Quotients}

Typically, every local section of our development assumes once and
for all the existence of one group domain \C{elt} to then manipulate 
different groups of this domain.
Nevertheless, there are situations where it is necessary to build 
new \C{finGroupType} structures. This is the case for example for
\emph{quotients}.
Let \C{H} and \C{K} be two groups in the same group universe,
the quotient \C{K/H} is a group under the condition that
\C{H} is \emph{normal} in \C{K}.
Of course, we could create a new group domain for each quotient, but we
can be slightly smarter noticing that given a group \C{H}, all the 
quotients of the form \C{K/H} share the same  group law, and the same unit.
The idea is then to have all the quotients groups \C{K/H} in a group domain \C{./H}.
The largest possible quotient is \C{N(H)/H}, where \C{N(H)} is the normaliser of \C{H} and
all the other quotients are subsets of this one.


In our formalisation, normality is defined as:
\begin{lstlisting}
Definition H $\triangleleft$ K := (H $\subset$ K) && (K $\subset$ (normaliser H)).
\end{lstlisting}
If \C{H $\,\triangleleft\,$ K}, \C{H}-left cosets and \C{H}-right cosets coincide
for every element of \C{K}. Hence, they are just called \emph{cosets}.
Once again, we carefully stick to first order predicates to take as
much benefit as possible from the canonical structure
mechanism. If necessary, side conditions are embedded inside
definitions by the mean of boolean tests. Like this, we avoid having
to add pre-conditions in the properties of these predicates to insure well-formedness.
The definition of cosets makes no restriction on its
arguments:

\begin{lstlisting}
Definition coset (A : setType elt) (x : elt) := 
      if (x \in (normaliser A)) then A :* x else A.
\end{lstlisting}

The set of cosets of an arbitrary set \lstinline[basicstyle=\footnotesize]+A+
is the image of the whole group domain by the \lstinline[basicstyle=\footnotesize]+coset+
operation. Here we define the associated sigma type:
\begin{lstlisting}
Definition cosets (A : setType elt):= iimage (coset A) elt.
Definition cosetType (A : setType elt):= eq_sig (cosets A).
\end{lstlisting}
\noindent where \lstinline[basicstyle=\footnotesize]+eq_sig+ builds the sigma type associated to a
\lstinline[basicstyle=\footnotesize]+set+. This
\lstinline[basicstyle=\footnotesize]+cosetType+ type can be equipped
with canonical structures of
\lstinline[basicstyle=\footnotesize]+eqType+ and
\lstinline[basicstyle=\footnotesize]+finType+ and elements of this type are intentional sets.

\noindent The \lstinline[basicstyle=\footnotesize]+quotient+ of two groups of the same group domain can
\emph{always} be defined:

\begin{lstlisting}
Definition  A/B := iimage (coset_of B) A.
\end{lstlisting}
\noindent where 
\lstinline[basicstyle=\footnotesize]+coset_of : elt -> (cosetType A)+ injects the
value of \lstinline[basicstyle=\footnotesize]+(coset A x)+ in
\lstinline[basicstyle=\footnotesize]+(cosetType A)+. Thanks to the internal boolean test in \C{coset},
\lstinline[basicstyle=\footnotesize]+A/B+ defines in fact $[A \cap N(B)]/B$.


When \lstinline[basicstyle=\footnotesize]+H+ is equipped with a group structure, we define group
operations on \lstinline[basicstyle=\footnotesize]+(cosetType H)+ thanks to the following properties:
\begin{lstlisting}
Lemma cosets_unit : H \in (cosets H).
Lemma cosets_mul : forall Hx Hy : cosetType H, (Hx spp Hy) \in (cosets H).
Lemma cosets_inv : forall Hx : cosetType H, (Hx :^-1) \in (cosets H).
\end{lstlisting}
\noindent where \lstinline[basicstyle=\footnotesize]+A :^-1+ denotes the image of a set
\lstinline[basicstyle=\footnotesize]+A+ by the
inverse operation. Group properties are provable for these operations:
we can define a canonical structure of group domain on
\lstinline[basicstyle=\footnotesize]+cosetType+, depending on an arbitrary group
object. Canonical structures of \emph{groups}, in this group domain,
are defined for every quotient of two group structures. A key point in
the readability of statements involving quotients is that the
\lstinline[basicstyle=\footnotesize]+./.+ notation is usable because it refers to a definition
independent of proofs; the type inference mechanism will
automatically find an associated group structure for this set when it
exists.

Defining quotients has also been a place where we had to rework 
our formalisation substantially using intensional sets instead of sets defined
by their characteristic function. In the library of finite group quotients,
there are two kinds of general results. The first one states
\emph{equalities} between quotients, like the theorems about the kernel of quotient
morphism. The second, often heavily relying on properties of the first kind, builds
isomorphisms between different groups, i.e. groups having distinct
carriers (and hence operations). For example, this is the case for the
so-called three fundamental isomorphism theorems. 
The initial version of the quotients was using sets defined by their characteristic function.
Having sets
for which function extensionality does not hold had forced us to
use setoid.
For theorems with types depending on setoid arguments, especially the ones stating
equalities, we had to add  one extensional equality condition per occurrence
of such a dependant type in the statement of the theorem in order to make these
theorems usable. The situation was even worse since, in order to apply one of these theorems.
 the user had to provide specific lemmas, proved before-hand, for each equality proof.
This was clearly unacceptable if quotients were to be used in further
formalisations.
Using intensional sets has simplified everything.

\subsection{Group Morphisms}
Group morphisms are functions between two group domains, which 
comply with the group laws of their domain and
codomain. Since we do not create one type per group, the notion of
morphism is parametrised by a group on which morphism properties
hold. The fundamental property of group morphisms is that they
preserve group structures under image and pre-image. 

To avoid having to use technical lemmas about the restriction of
morphism domains, we want the image and preimage of groups by morphism to
have a \emph{canonical} structure of group. Thus, the values of a given
function alone should be enough to determine the largest group on which this
function may be seen as a morphism.

We have embedded the domain of a morphism inside its computational definition
by giving a default unit value outside the group
where the morphism properties are supposed to hold.
Now, the problem is to compute back the domain of a morphism candidate from
 its values, identifying the kernel among the set of elements mapped to the unit:
\begin{lstlisting}
Definition ker (f: elt -> elt') := {x:elt $\subset$ {y: elt, f (x * y) == f y}}.
\end{lstlisting}
\noindent which can be equipped with a canonical group structure. Morphism
domains are defined as:
\begin{lstlisting}
Definition mdom (f: elt -> elt') := ker $\cup$ {x, f x != 1}.
\end{lstlisting}
Morphisms are defined by the following structure:
\begin{lstlisting}
Structure morphism : Type := Morphism {
  mfun :> elt -> elt';
  group_set_mdom : group_set (mdom mfun);
  morphM : forall x y, 
   (mfun x) \in mdom => (mfun y) \in mdom => mfun (x * y) = mfun x * mfun y
}.
\end{lstlisting}

An isomorphism is a morphism having a trivial kernel. Restricting a
morphism is simply done by giving the default unit value outside its intended domain.
This operation is a canonical morphism construction.
Morphisms and quotients are involved in the universal property of morphism
factorisation. For any function between group domains, we define a
quotient function by:

\begin{lstlisting}
Definition fquo H (f : elt -> elt'):=
  if H $\subset$ (ker f) then fun (Hx : cosetType H) => f (repr Hx)
  else fun (Hx : cosetType H) => 1.
\end{lstlisting}
where \lstinline[basicstyle=\footnotesize]+repr+ picks a
representative in any set of a
\lstinline[basicstyle=\footnotesize]+finGroupType+.
Given any morphism, its quotient function defines an isomorphism
between the quotient of its domain by its kernel and the image of the
initial morphism.

This definition of morphism has been motivated by the formal proofs of the
three fundamental isomorphism theorems. The goal was to eliminate
any proof dependency which cannot be resolved by the type inference system with the help
of canonical structures. The result is that statements are much more readable
and formal proofs much easier.
For instance, the third fundamental isomorphism theorem follows directly
from the three lemmas below, because the
function~\lstinline[basicstyle=\footnotesize]+f3+ is \emph{canonically}
a morphism.

\begin{lstlisting}
Hypothesis sHK : H $\subset$ K.
Hypothesis nHG : H $\triangleleft$ G.
Hypothesis nKG : K $\triangleleft$ G.

Let f3 := (fquo (fquo (coset K)).

Lemma mdom_f3 : mdom f3 $\subset$ (G / H) / (K / H).
Lemma im_f3 : iimage f3 = G / K.
Lemma f3_ker :  (ker f3) = {1}.
\end{lstlisting}

\section{Standard theorems of group theory}

In order to evaluate how practical our definitions of groups, cosets and quotients were, we 
have started formalising some standard results of group theory. In this section,
we present three of them: the Sylow theorems, the Frobenius lemma and the Cauchy-Frobenius lemma. 
The Sylow theorems are central in group theory. The Frobenius lemma gives a nice property
of the elements of a group of a given order. Finally the Cauchy-Frobenius 
lemma, also called the Burnside counting lemma, applies directly to 
enumeration problems. 
Our main source of inspiration for these proofs 
was some lecture notes
on group theory by Constantine~\cite{Const}.

\subsection{The Sylow theorems}

The first Sylow theorem asserts the existence of a subgroup \C{H} of
\C{K} of cardinality $\C{p} ^n$,
for every prime $\C{p}$ such that $\C{card}(\C{K}) = \C{p}^n
s$ and $\C{p}$ does not divide \C{s}. For any two groups \C{H} and
\C{K} of the same group domain, we define and prove :
\begin{lstlisting}
Definition sylow K p H := H $\subset$ K && card H == p$^{\log_{\rm{p}} \rm (card\ K)}$.

Theorem sylow$_1$: forall K p, exists H, sylow K p H.
\end{lstlisting}
The first definition captures the property of \C{H} being a
\C{p}-Sylow subgroup of \mbox{\C{K}.}
The expression $\log_{\rm{p}} \C{(card K)}$ computes the maximum value
of \C{i} such that
$\C{p}^i$ divides the cardinality of \C{K} when \C{p} is prime.
This theorem has already been formalised by Kamm�ller and Paulson~\cite{kam},
based a proof due to Wielandt~\cite{Wielandt}.
Our proof is slightly different and intensively uses group actions on sets. 
Given a group domain \C{G} and a finite type \C{S}, actions are defined
by the following structure:
\begin{lstlisting}
Structure action : Type := Action {
  act_f :> S -> G -> S;
  act_1 : forall x, act_f x 1 = x;
  act_morph : forall (x y : G) z, act_f z (x * y) = act_f (act_f z x) y 
}.
\end{lstlisting}
Note that we take advantage of our finite setting to replace the 
usual bijectivity of the action by the simpler property that
the unit acts trivially.

A complete account of our proof is given in~\cite{rt_sylow}.
The proof works by induction on $n$ showing that there exists a subgroup of order $\C{p}^i$ for all $0 < i \le n$.
The base case is Cauchy theorem, which asserts that a group \C{K}
has an element of order
\C{p} for each prime divisor of the cardinality of the group \C{K}.
Our proof is simpler than the combinatorial argument used in~\cite{kam}, which
hinged on properties of the binomial. We first build the set $U$ such that  $ U = \{ (k_1, \dots, k_{{\rm{p}}}) \,\,|
\,\,
k_i \in \C{K} \, \textrm{and}\, \prod_{i=1}^{i={\rm{p}}} k_i = 1\}$. We have that ${\rm card}(U) = {\rm card}(\C{K})^{{\rm{p}}-1}$.
We then define the action of the additive group $\mathbb{Z}/{\rm{p}}\mathbb{Z}$ that acts on $U$ as
$$ n \quad \longmapsto \quad (k_1,\dots, k_{{\rm{p}}}) \mapsto (k_{n \,
  mod \, {\rm{p}} + 1},\dots, k_{(n + {\rm{p}} -1) \, mod\, {\rm{p}} + 1})$$
Note that defining this action is straightforward since \C{p}-tuples are graphs of functions
whose domain is \C{(ordinal p)}.

Now, we consider the set $S_0$ of the elements of $U$ whose orbits by the 
action are a singleton. $S_0$ is composed of the elements $(k,\dots,k)$ such that $k\in \C{K}$ and $k^p=1$.
A consequence of the orbit stabiliser theorem tells  us that \C{p} divides
the cardinality of $S_0$. As $S_0$ is non-empty ($(1,\dots,1)$ belongs to $S_0$), there
exists at least one $k\neq 1 $, such that $(k,\dots,k)$ belongs to $S_0$. The order of $k$ is then $\C{p}$.

In a similar way, in the inductive case, we suppose that there is a subgroup \C{H} of order $\C{p}^i$,
we consider ${\it N_{{\rm{K}}}(\C{H})/\C{H}}$ the quotient of the normaliser of \C{H} in \C{K} 
by \C{H}. We act with \C{H} on the left cosets
of \C{H} by left translation:
$$ g \quad \longmapsto \quad h\C{H} \mapsto (gh)\C{H} $$
and consider the set $S_0$ of the left coset of \C{H} whose orbits by the 
action are a singleton. The elements of $S_0$ are exactly the elements of $N_{{\rm{K}}}(\C{H})/\C{H}$.
Again, applying the orbit stabiliser theorem, we can deduce that \C{p} divides the cardinality of $S_0$ so
there exists an element $k$ of order \C{p} in $S_0$ by Cauchy theorem. If we consider  \C{H$\scriptstyle'$}, the pre-image
by the quotient operation of the cyclic group generated by $k$, its cardinality is $\C{p}^{i+1}$. 

We have also formalised the second and third Sylow theorems. The second theorem states that any
two \C{p}-Sylow subgroups \C{H}$_1$ and \C{H}$_2$ are conjugate. This is proved
acting with \C{H}$_1$ on the left coset of \C{H}$_2$. The third theorem 
 states that the number of \C{p}-Sylow subgroups divides the cardinality
of \C{K} and is equal to 1 modulo \C{p}. The third theorem is proved by acting by 
conjugation on the sets of all \C{p}-Sylow subgroups. 

\subsection{The Frobenius lemma}

Given an element $a$ of a group $G$, \C{(cyclic $a$)} builds the
cyclic group generated by $a$. When proving properties of cyclic groups,
we use the characteristic property of the \C{cyclic} function.
\begin{lstlisting}[fontadjust]
Lemma cyclicP: forall a b, reflect (exists n, a^n == b) (cyclic a b). 
\end{lstlisting}
The order of an element is then defined as the cardinality of its associated cyclic
group. The Frobenius lemma states that given a number $n$ that divides the cardinality of a 
group \C{K}, the number of elements whose order divides $n$ is a multiple
of $n$. In our formalisation, this gives
\begin{lstlisting}
Theorem frobenius: forall K n, n | (card K)  ->   n | (card {z:K,  (orderg z) | n}).
\end{lstlisting}

The proof is rather technical and has intensively tested our library on cyclic groups.
For example, as we are counting the number of elements of a given order, we need
to know the number of generators of a cyclic group. This is given by a theorem of
our library
\begin{lstlisting}
Lemma gphi_gen: forall a, gphi (orderg a) = card (generator (cyclic a)).
\end{lstlisting}
where \C{gphi} is the Euler function.

\subsection{The Cauchy-Frobenius lemma}


Let $G$ a group acting on a set $S$. For each $g$ in $G$, let $F_g$ be the set
of elements in $S$ fixed by $g$, and $t$ the number of orbits of $G$ on
$S$, then $t$ is equal to the average number of points left fixed by
each element of $G$:
$$ t = \frac {1} {|G|} \sum_{g\in G} |F_g|$$
\noindent
To prove this lemma, we consider B,  subset of the cartesian product $G \times S$ containing the  pairs $(g,x)$ such that  $g(x) = x$.
We use two ways to evaluate the cardinality of B, first  by fixing the first
component: 
$|B| = \sum_{g\in G}\vert F_g\vert$, then by fixing the second component:
$|B|=\sum_{ x\in S}\vert G_x\vert$ where $G_x$ is the stabiliser of $x$ in $G$.
Then, when sorting the right hand-side of the second equality by
orbits we obtain that 
$|B|=|G{x_1}||G_{x_1}| +|G{x_2}| |G_{x_2}| +\cdots +|G{x_t}| |G_{x_t}|$
the $x_i$ being representatives of the orbit $G{x_i}$.
Applying the Lagrange theorem on the stabiliser of  $x_i$ in G (the
subgroup $G_{x_i}$), we obtain that for each orbit:
$|G{x_i}| |G_{x_i}| = |G|$ and we deduce that  $|B|= t |G| = \sum_{g\in G} |F_g|$.

This lemma is a special case of the powerful P�lya method, but it
already has significant applications in combinatorial counting problems. 
To illustrate this, we have formally shown that there are 55 distinct
ways of colouring with 4 colours the vertices of a square up to isometry.  
This is done by instantiating a more general theorem that tells that the number of 
ways of colouring with $n$ colours
is \mbox {$(n^4 + 2n^3 + 3n^2 + 2n)/8$}. This last theorem is a direct application
of the Cauchy-Frobenius lemma. The encoding of the problem is
the following:
\begin{lstlisting}
 Definition square := ordinal 4.
 Definition colour := ordinal n.
 Definition colouring := fgraphType square colour.
\end{lstlisting}
Vertices are represented by the set $\{0,1,2,3\}$, colours 
by the set $\{0,1,\dots,n-1\}$ and colouring by functions
from vertices to colours. The set of isometries is a subset of the permutations
of \C{square} that preserve the geometry of the square. In our case,
we use the characteristic condition that {\it the images of two opposite vertices
remain opposite}.
\begin{lstlisting}
 Definition isometry := {p : perm square, forall i, p (opp i) = opp (p i)}.
\end{lstlisting}
where % \lstinline[basicstyle=\footnotesize]+$\forall_f$+ is
% the decidable universal quantification on finite types,
\C{perm square} the permutation group and
 \C{opp}  the function that returns the opposite of a vertex. 
We get that the isometries is a subgroup of the permutations,
since the property of conserving opposite vertices is stable
by composition and the identity obviously preserve opposite vertices.

The action of an isometry $p$ on a colouring $c$ returns the colouring $i \mapsto c(p(i))$.
Each set of
identical coloured squares corresponds to an orbit of this
action. To apply Cauchy-Frobenius, we first need to give an extensional
definition of the isometries, i.e., there are 8 isometries: the identity, 
the 3 rotations of $\pi /2$, $\pi$ and $3\pi /2$, the vertical symmetry, 
the horizontal symmetry and the 2 symmetries about the diagonals.
Second, we have to count the elements left fixed by each  isometry.

The proofs of the  three theorems presented in this section manipulate many
of the base concepts defined in  our formalisation. They have been particularly 
important to give us feedback on how practical our definitions were.

\section{Conclusion}


To our knowledge, what is presented in this paper is already one of the most
complete formalisations of finite group theory. 
We cover almost all the material that can be found in an introductory course on 
group theory. Very few standard results like the simplicity
of the alternating group are still missing, but should be formalised very soon.
The only similar effort but in set theory
can be found in the Mizar system~\cite{Mizar}. Theorems like the ones
presented in Section 4 are missing from the Mizar formalisation.

We have deliberately specialised our formalisation to \emph{finite}
group theory: finite groups are \emph{not} obtained as a sub-class of
generic groups.
% Such a generic theory of groups is needed for the
% character theory part of the Feit-Thompson proof, hence we plan to
% elaborate this as a separated part of the development.
This design choice
is consistent with the usual presentation of group theory in the literature.
Imposing the inheritance of finite group
theory from generic group theory would be somehow artificial since they share
little results, all of them being trivial ones, hence it would not
help formalising proofs but create pointless delicate subtyping issues.  

Getting the definitions right is one of the most difficult aspects of
formalising mathematics. The problem is not much in capturing the 
semantics of each individual construct but rather in having all the concepts 
working together well. Group theory has been no exception in that respect.
It took much trial and error to arrive at the definitions 
presented in this paper. The fact that we were able to get 
results like the ones presented in Section 4 relatively
easily makes us confident 
that our base is robust enough to proceed to further formalisations.

Using \ssr{} has been a key aspect to our formal
development. Decidable types and a substantial use of rewriting for our proofs
give a `classical' flavour to our development that is more familiar
to what can be found in provers like {\sc Isabelle}~\cite{Isabelle} or
{\sc Hol}~\cite{Hol} than what is usually done in \Coq. Moreover, our novel use of canonical structures allows us to reconcile the convenience of
set-theoretic statements with the expressiveness of dependent types, by harnessing the automation power of type inference. We think that
this combination makes the \Coq{} system a powerful environment for
the formalisation of such algebraic theories.




An indication of the conciseness of our proof scripts
is given by the following figure. The standard library of \Coq{} contains
7000 objects (definitions + theorems) for  93000 lines of code, this makes a ratio of 
13 lines per object. The base library of \ssr{} plus our library for groups contains
1980 objects for 14400 lines, this makes a ratio of 7 lines per object.




\bibliographystyle{plain}
\bibliography{main}

\end{document}

