\section{Changes}

\subsection{\ssr{} version 1.3}
All changes are retrocompatible extensions but for:
\begin{itemize}
\item Occurrences in the type family switch now refer only to the goal, while
      before they used to refer also to the types in the abstractions of the 
      predicate used by the eliminator. This bug used to affect lemmas like 
      \C{boolP}. See the relative comments in \C{ssrbool.v}.
\item Clear switches can only mention existing hypothesis and
      otherwise fail. This can in particular affect intro patterns
      simultaneously applied to several goals.
      % commit: 2686
\item A bug in the \C{rewrite} tactic allowed to
      instantiate existential metavariables occurring in the goal.
      This is not the case any longer (see section~\ref{ssec:rewcaveats}).
\item The \C{fold} and \C{unfold} $\N{r-items}$ for \C{rewrite} used to 
      fail silently when used in combination with a $\N{r-pattern}$ matching no
      goal subterm. They now fail. The old behaviour can be obtained using
      the \C{?} multiplier (see section~\ref{ssec:extrw}).
\item Coq 8.2 users with a statically linked toplevel must comment out the\\
      \C{Declare ML Module "ssreflect".}\\
      line at the beginning of \C{ssreflect.v} to compile the 1.3 library.
\end{itemize}
New features:
\begin{itemize}
\item Contextual \C{rewrite} patterns.
      The context surrounding the redex can now be used to specify which
      redex occurrences should be rewritten (see section~\ref{ssec:rewp}).\\
      \C{rewrite [in X in _ = X]addnC.}
      % commit: 2690, 2689, 2718, 2733
\item Proof irrelevant interpretation of goals with existential metavariables.
      Goals containing an existential metavariable of sort \C{Prop} are
      generalized over it, and a new goal for the missing subproof is 
      generated (see page~\pageref{sssec:apply} and 
      section~\ref{ssec:rewcaveats}).\\
      \C{apply: (ex_intro _ (@Ordinal _ y _)).}\\
      \C{rewrite insubT.}
      % commit: 2553, 2544, 2543, 2733
\item Views are now part of $\N{i-pattern}$ and can thus be used
      inside intro patterns (see section~\ref{ssec:intro}).\\
      \C{move=> a b /andP [Ha Hb].}
      % commit: 2720
\item Multiple views for \C{move}, \C{move ... in} and \C{apply} 
      (see section~\ref{ssec:multiview}).\\
      \C{move/v1/v2/v3.}\\
      \C{move/v1/v2/v3 in H.}\\
      \C{apply/v1/v2/v3.}
      % commit: 2720
\item \C{have} and \C{suff} idiom with view (see section~\ref{sssec:hypview}).
\begin{lstlisting}
  Lemma |*test*| (a b : bool) (pab : a && b) : b.
  have {pab} /= /andP [pa ->] // : true && (a && b) := pab.
\end{lstlisting}
      % commit: 2726
\item \C{have suff}, \C{suff have} and \C{wlog suff} forward reasoning 
      tactics (see section~\ref{ssec:struct}).\\
      \C{have suff H : P.}
      % commit: 2633
\item Binders support in \C{have} (see section~\ref{sssec:have}).\\
      \C{have H x y (r : R x y) : P x -> Q y.}
      % commit: 2633
\item Deferred clear switches. Clears are deferred to the end of the 
      intro pattern. In the meanwhile, cleared variables are still
      part of the context, thus the goal can mention them, but are 
      renamed to non accessible dummy names (see section~\ref{ssec:intro}).\\
      \C{suff: G \\x H = K; first case/dprodP=> \{G H\} [[G H -> -> defK]].}
      % commit: 2660
\item Relaxed alternation condition in intro patterns. The
      $\N{i-item}$ grammar rule is simplified (see section~\ref{ssec:intro}).\\
       \C{move=> a \{H\} /= \{H1\} // b c /= \{H2\}.}
      % commit: 2713
\item Occurrence selection for \C{->} and \C{<-} intro pattern
       (see section~\ref{ssec:intro}).\\
      \C{move=> a b H \{2\}->.}
      % commit: 2714
\item Modifiers for the discharging '\C{:}' and \C{in} tactical to override 
      the default behaviour when dealing with local definitions (let-in):
      \C{@f} forces the body of \C{f} to be kept, \C{(f)} forces the body of
      \C{f} to be dropped  (see sections~\ref{ssec:discharge} 
      and~\ref{ssec:gloc}).\\
      \C{move: x y @f z.}\\
      \C{rewrite rule in (f) $\;\;$H.}
      %commit: 2659, 2710
\item Type family switch in \C{elim} and \C{case}
      can contain patterns with occurrence switch
      (see section~\ref{ssec:typefam}).\\
      \C{case: \{2\}(_ == x) / eqP.}
      % commit: 2593, 2598, 2539, 2538, 2527, 2529
\item Generic second order predicate support for \C{elim}
      (see section~\ref{sec:views}).\\
      \C{elim/big\_prop: _}
      % commit: 2767
\item The \C{congr} tactic now also works on products (see
      section~\ref{ssec:congr}).
\begin{lstlisting}
  Lemma |*test*| x (H : P x) : P y. 
  congr (P _): H.
\end{lstlisting}
      % commit: 2608
\item Selectors now support $\mathcal{L}$-tac variables 
      (see section~\ref{ssec:select}).\\
      \C{let n := 3 in tac; first n last.}
      % commit: 2725
\item Deprecated use of \C{Import Prenex Implicits} directive. 
      It must be replaced with the standard \Coq{} \C{Unset Printing
        Implicit Defensive} vernacular command.
\item New synonym \C{Canonical} for \C{Canonical Structure}.
\end{itemize}
\subsection{\ssr{} version 1.4}
New features:
\begin{itemize}
\item User definable recurrent contexts (see section~\ref{ssec:rewp}).\\
      \C{Notation RHS := (X in _ = X)\%pattern} 
\item Contextual patterns in 
      \C{set} and `\C{:}' (see section~\ref{ssec:rewp}).\\
      \C{set t := (a + _ in RHS)}
\item NO-OP intro pattern (see section~\ref{ssec:intro}).\\
      \C{move=> /eqP-H /fooP-/barP}
\item \C{if ... isn't }notation (see section~\ref{ssec:patcond}).\\
      \C{if x isn't Some y then simple else complex y}
\end{itemize}
\subsection{\ssr{} version 1.5}
Incompatibilities:
\begin{itemize}
\item The \C{have} tactic now performs type classes resolution.  The old
	behavior can be restored with \C{Set SsrHave NoTCResolution}
\end{itemize}
Fixes:
\begin{itemize}
\item The \C{let foo := type of t in} syntax of standard \C{Ltac} has
	been made compatible with \ssr{} and can be freely used even if
	the \ssr{} plugin is loaded
\end{itemize}
New features:
\begin{itemize}
\item Generalizations supported in have (see section~\ref{ssec:struct}).\\
      \C{generally have hx2px, pa : a ha / P a.}
\item Renaming and patterns in wlog (see section~\ref{ssec:struct} and
  page \pageref{par:advancedgen}).\\
      \C{wlog H : (n := m)$\;$ (x := m + _)$\;$ / T x}.\\
      \C{wlog H : (n := m)$\;$ (@ldef := secdef m)$\;$ / T x}.
\item Renaming, patterns and clear switches in \C{in}
      tactical (see section~\ref{ssec:gloc}).\\
      \C{... in H1 \{H2\} (n := m).}
\item Handling of type classes in \C{have}
	(see page~\pageref{ssec:havetcresolution}).\\
	\C{have foo : ty. (* TC inference for ty *)}\\
	\C{have foo : ty := . (* no TC inference for ty *)}\\
	\C{have foo : ty := t. (* no TC inference for ty and t *)}\\
	\C{have foo := t. (* no TC inference for t *)}
\item Transparent flag for \C{have} to generate a \C{let in} context entry
	(see page~\pageref{sec:havetransparent}).\\
        \C{have @i : 'I\_n by apply: (Sub m); auto.} 
\item Intro pattern \C{[: foo bar ]} to create abstract variables
	(see page~\pageref{ssec:introabstract}).
\item Tactic \C{abstract:} to assign an abstract variable
	(see page~\pageref{ssec:abstract}).\\
  \C{have [: blurb ] @i : 'I\_n by apply: (Sub m); abstract: blurb; auto.}\\
  \C{have [: blurb ] i : 'I\_n := Sub m  blurb; first by auto.}

\end{itemize}
